%% Generated by Sphinx.
\def\sphinxdocclass{jupyterBook}
\documentclass[letterpaper,10pt,english]{jupyterBook}
\ifdefined\pdfpxdimen
   \let\sphinxpxdimen\pdfpxdimen\else\newdimen\sphinxpxdimen
\fi \sphinxpxdimen=.75bp\relax
%% turn off hyperref patch of \index as sphinx.xdy xindy module takes care of
%% suitable \hyperpage mark-up, working around hyperref-xindy incompatibility
\PassOptionsToPackage{hyperindex=false}{hyperref}
%% memoir class requires extra handling
\makeatletter\@ifclassloaded{memoir}
{\ifdefined\memhyperindexfalse\memhyperindexfalse\fi}{}\makeatother

\PassOptionsToPackage{warn}{textcomp}

\catcode`^^^^00a0\active\protected\def^^^^00a0{\leavevmode\nobreak\ }
\usepackage{cmap}
\usepackage{fontspec}
\defaultfontfeatures[\rmfamily,\sffamily,\ttfamily]{}
\usepackage{amsmath,amssymb,amstext}
\usepackage{polyglossia}
\setmainlanguage{english}



\setmainfont{FreeSerif}[
  Extension      = .otf,
  UprightFont    = *,
  ItalicFont     = *Italic,
  BoldFont       = *Bold,
  BoldItalicFont = *BoldItalic
]
\setsansfont{FreeSans}[
  Extension      = .otf,
  UprightFont    = *,
  ItalicFont     = *Oblique,
  BoldFont       = *Bold,
  BoldItalicFont = *BoldOblique,
]
\setmonofont{FreeMono}[
  Extension      = .otf,
  UprightFont    = *,
  ItalicFont     = *Oblique,
  BoldFont       = *Bold,
  BoldItalicFont = *BoldOblique,
]


\usepackage[Bjarne]{fncychap}
\usepackage[,numfigreset=1,mathnumfig]{sphinx}

\fvset{fontsize=\small}
\usepackage{geometry}


% Include hyperref last.
\usepackage{hyperref}
% Fix anchor placement for figures with captions.
\usepackage{hypcap}% it must be loaded after hyperref.
% Set up styles of URL: it should be placed after hyperref.
\urlstyle{same}


\usepackage{sphinxmessages}



        % Start of preamble defined in sphinx-jupyterbook-latex %
         \usepackage[Latin,Greek]{ucharclasses}
        \usepackage{unicode-math}
        % fixing title of the toc
        \addto\captionsenglish{\renewcommand{\contentsname}{Contents}}
        \hypersetup{
            pdfencoding=auto,
            psdextra
        }
        % End of preamble defined in sphinx-jupyterbook-latex %
        

\title{Simple Chess}
\date{Jul 29, 2021}
\release{}
\author{Sabina Sara Pfister}
\newcommand{\sphinxlogo}{\vbox{}}
\renewcommand{\releasename}{}
\makeindex
\begin{document}

\pagestyle{empty}
\sphinxmaketitle
\pagestyle{plain}
\sphinxtableofcontents
\pagestyle{normal}
\phantomsection\label{\detokenize{intro::doc}}


\sphinxAtStartPar
I’m an avarage chess player.

\sphinxAtStartPar
Nowadays chess engine on protable devices easely defeat human play. This is not surprising, since computers have the capability to evaluate thousend of move combinations in few seconds, while the human mind can compute few moves in avance at best.

\sphinxAtStartPar
After introducing the basic rules of chess, and some examples at different stages of the game, I’ll introduce the concpet of using Machine Intelligence to understand game patterns, and how to use such computational capabilities to empower and teach the human mind to reach its potential. For those skeptical on the possiblity of using machines to enhance human learning, we provide compelling evidence for the power of mind\sphinxhyphen{}machine interaction with a short analysis of the history of chess.


\chapter{The Chess Board}
\label{\detokenize{chapter1:the-chess-board}}\label{\detokenize{chapter1::doc}}
\begin{sphinxVerbatim}[commandchars=\\\{\}]
\PYGZlt{}IPython.core.display.HTML object\PYGZgt{}
\end{sphinxVerbatim}

\sphinxAtStartPar
Chess, as we know it today, was born out of the Indian game chaturanga in the 7th century, and slowly spread to Asia and Persia. The game was inspired by battle, with chess pieces named according to the army divisions of the time. Once chess reached Europe in the 15th century, the pieces names were formalized to \sphinxstyleemphasis{pawn}, \sphinxstyleemphasis{knight}, \sphinxstyleemphasis{bishop}, \sphinxstyleemphasis{rook} and \sphinxstyleemphasis{queen} or \sphinxstyleemphasis{king} to reflect the composition of modern armies.

\sphinxAtStartPar
Two opponent players, white and black, are assigned each a total of 16 pieces positioned on the battlefield, a 8x8 grid called \sphinxstyleemphasis{chessboard}. Pieces are generally moved into positions where they can capture other pieces or defend their own pieces. The games ends when the opponent king is captured, a move termed \sphinxstyleemphasis{checkmate}.

\sphinxAtStartPar
Below you can see a table of the different pieces, end their names throughout chess history.

\begin{sphinxVerbatim}[commandchars=\\\{\}]
\PYGZlt{}IPython.core.display.HTML object\PYGZgt{}
\end{sphinxVerbatim}


\section{The Chess Board}
\label{\detokenize{chapter1:id1}}
\sphinxAtStartPar
Chess is a two\sphinxhyphen{}player strategy board game played on a chessboard, a checkered gameboard with 64 squares arranged in an 8×8 grid, where each square is labaled by its row(a\sphinxhyphen{}h) and column position (1\sphinxhyphen{}8). Each player begins with 16 pieces, either black or white, arranged at the top and bottom of the chessboard.

\begin{sphinxVerbatim}[commandchars=\\\{\}]
\PYGZlt{}IPython.core.display.SVG object\PYGZgt{}
\end{sphinxVerbatim}

\sphinxAtStartPar
\sphinxstylestrong{Question:} Can you locate the inital position of each piece for white and black?

\begin{sphinxadmonition}{note}{Note:}
\sphinxAtStartPar
Here’s my figure:
{\color{red}\bfseries{}:glue:figure:`sorted\_means\_fig`}
\end{sphinxadmonition}


\section{The White Pawn}
\label{\detokenize{chapter1:the-white-pawn}}
\sphinxAtStartPar
The pawn is the most numerous piece in the game of chess, and in most circumstances, also the weakest. Historically it represents the army infantry, in particular armed peasants or pikemens. Normally white pawns move by advancing a single square upward. However, the first time a pawn is moved, it has the option of advancing two squares. Unlike other pieces, the pawn does not capture in the same direction as it moves, instead it captures diagonally forward one square to the left or right. A pawn that advances all the way to the opposite side of the board (the opposing player’s first rank) is promoted to another piece of that player’s choice: a \sphinxstyleemphasis{queen}, \sphinxstyleemphasis{rook}, \sphinxstyleemphasis{bishop}, or \sphinxstyleemphasis{knight} of the same color.

\begin{sphinxVerbatim}[commandchars=\\\{\}]
\PYGZlt{}IPython.core.display.SVG object\PYGZgt{}
\end{sphinxVerbatim}

\sphinxAtStartPar
\sphinxstylestrong{Question:} Can you trace all the possible moves that white can take? And which piece can be promoted?







\renewcommand{\indexname}{Index}
\printindex
\end{document}